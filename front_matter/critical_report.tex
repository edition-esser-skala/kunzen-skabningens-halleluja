\documentclass[parskip=full]{scrreprt}

\RedeclareSectionCommand[pagestyle=plain,afterskip=10pt plus 1pt]{chapter}
\setkomafont{chapter}{\mdseries\headingfont\fontsize{40}{40}\selectfont\color{black!80}}
\setkomafont{pageheadfoot}{\normalsize}

\def\pnumbox#1{#1\hspace*{9cm}}
\DeclareTOCStyleEntry[
  indent=0pt,
  entrynumberformat=\textcolor{oldred},
  numwidth=2em,
  linefill=\hfill,
  pagenumberbox=\pnumbox,
  pagenumberformat=\itshape
]{tocline}{section}

\usepackage[english]{babel}

\usepackage{fontspec}

\setmainfont{Source Sans Pro}[
  ItalicFont = Source Sans Pro Italic,
  BoldFont = Source Sans Pro Bold,
  BoldItalicFont = Source Sans Pro Bold Italic,
  FontFace = {lt}{n}{Source Sans Pro Light},
  FontFace = {lt}{it}{Source Sans Pro Light Italic},
  FontFace = {sb}{n}{Source Sans Pro Semibold},
  FontFace = {sb}{it}{Source Sans Pro Semibold Italic},
  Numbers = Proportional,
  Ligatures = Common
]
\DeclareRobustCommand{\ltseries}{\fontseries{lt}\selectfont}
\DeclareRobustCommand{\sbseries}{\fontseries{sb}\selectfont}
\DeclareTextFontCommand{\textlt}{\ltseries}
\DeclareTextFontCommand{\textsb}{\sbseries}

\newfontfamily\headingfont{Fredericka the Great}

\usepackage[pass]{geometry}
\newgeometry{twoside,inner=20mm,outer=40mm,top=20mm,bottom=40mm}

\usepackage{url}
\urlstyle{same}

\usepackage{microtype}
\microtypesetup{verbose=silent}

\usepackage{booktabs,array,longtable}
\newcolumntype{L}[1]{>{\raggedright\let\newline\\\arraybackslash\hspace{0pt}}p{#1}}

\usepackage{graphicx}

\usepackage{xcolor}
\definecolor{oldred}{rgb}{.8313,0,0}

\usepackage{pdfpages}

\usepackage{scrlayer-scrpage}
\pagestyle{scrheadings}
\clearscrheadfoot
\cfoot[\thepage]{\thepage}
\pagenumbering{roman}


\makeatletter

\newcommand\fancytitlehead{
	\headingfont%
	\fontsize{80}{80}\selectfont\textcolor{black!80}{\@lastname.}\\[15pt]%
	\fontsize{60}{60}\selectfont\@ifundefined{@shorttitle}{\@title}{\@shorttitle}.%
}


\def\firstname#1{\def\@firstname{#1}}
\def\lastname#1{\def\@lastname{#1}}
\def\shorttitle#1{\def\@shorttitle{#1}}
\def\namesuffix#1{\def\@namesuffix{#1}}
\def\instrumentation#1{\def\@instrumentation{#1}}
\def\parts#1{\def\@parts{#1}}

\firstname{\relax}
\lastname{\relax}
\namesuffix{\relax}
\instrumentation{\relax}
\parts{\relax}


\def\maketitle{%
\begin{titlepage}%
	\Large%
	{\@titlehead}%
	\vfill%
	{\strut\@firstname}\\%
	{\sbseries\color{oldred}\strut\@lastname}\\%
	{\strut\@namesuffix}%
	\vfill%
	{\sbseries\@title}\\%
	{\@subtitle}\\[\baselineskip]%
	{\itshape\@instrumentation}%
	\vfill%
	{\itshape\@parts}\hspace*{\fill}\raisebox{0pt}[0pt][0pt]{\includegraphics{ees_logo}}%
\end{titlepage}%
}


\newif\iftemplate\templatetrue
\newif\ifprintreport\printreportfalse
\directlua{
scores = {
  cl1 = "Clarinetto I",
  cl2 = "Clarinetto II",
  fag1 = "Fagotto I",
  fag2 = "Fagotto II",
  ottoni = "Corno I, II\string\\newline Tromba I, II\string\\newline Timpani",
  trb123 = "Trombone I, II, III",
  vl1 = "Violino I",
  vl2 = "Violino II",
  vla = "Viola",
  coro = "Coro",
  org = "Organo",
  b = "Bassi",
  full_score = "Full Score"
}

last_arg = arg[\string#arg]
texio.write("Last argument: " .. last_arg)
if not (scores[last_arg] == nil) then
  tex.print("\string\\def\string\\lypdfname{" .. last_arg .. "}")
  tex.print("\string\\parts{" .. scores[last_arg] .. "}")
  if (last_arg == "full_score") then
    tex.print("\string\\printreporttrue")
  end
end
}

\@ifundefined{lypdfname}{%
  \templatefalse
  \printreporttrue
  \parts{Draft}
}{\templatetrue}

\makeatother






\begin{document}

\titlehead{\fancytitlehead}
\firstname{Friedrich Ludwig Æmilius}
\lastname{Kunzen}
\title{Skabningens Halleluja}
\shorttitle{Skabningens\\Halleluja}
\subtitle{Das Halleluja der Schöpfung\\(DK-Kk mu 6506.1133)}
\instrumentation{2 S, A, T, B (coro), 2 fl, 2 ob, 2 cl, 2 fag, 2 cor, 2 tr, 3 trb, timp,\\vl solo, 2 vl, vla, b}
\maketitle


\thispagestyle{empty}

\vspace*{\fill}

\raisebox{-4mm}{\includegraphics{byncsaeu}}\hspace*{1em}Wolfgang Esser-Skala, 2020

© 2020 by Wolfgang Esser-Skala. This edition is licensed under the Creative Commons Attribution-NonCommercial-ShareAlike 4.0 International License. To view a copy of this license, visit \url{http://creativecommons.org/licenses/by-nc-sa/4.0/}. 

Music engraving by LilyPond 2.18.0 (\url{http://www.lilypond.org}).\\
Front matter typeset with Source Sans Pro and Fredericka the Great.

\textit{First version, April 2020}

\vspace*{2cm}

\ifprintreport
\chapter*{Critical Report.}

This edition bases upon a manuscript in the Det Kongelige Bibliotek på Slotsholmen – Den Sorte Diamant. The digital version of the manuscript is available at \url{http://img.kb.dk/ma/kunzen/SkaHal_m.pdf} (siglum mu 6506.1133).

In general, this edition closely follows the manuscript. Any changes that were introduced by the editor are indicated by italic type (dynamics and directions), parentheses (expressive marks) or dashes (slurs and ties). Accidentals are used according to modern conventions. Asterisks denote changes that are clarified in the detailed remarks below.

\bigskip

\footnotetext[1]{Abbreviations: A, alto; b, violoncello and bass; B, basso; cl, clarinet; cor, horn; fag, bassoon; ob, oboe; Ms, manuscript; r, rest; S, soprano; T, tenore; timp, timpani; tr, trumpet; trb, trombone; vl, violin; vla, viola.}

\begin{longtable}{lll L{10cm}}
	\toprule
	\itshape Mov. & \itshape Bar & \itshape Staff & \itshape Note \\
	\midrule \endhead
	
	\bottomrule
\end{longtable}


This edition has been compiled and checked with utmost diligence. Nevertheless, errors and mistakes cannot be totally excluded. Please report any errors and mistakes to \url{wolfgang@esser-skala.at} or create an issue or pull request on the edition’s GitHub page (https://github.com/skafdasschaf/kunzen-skabningens-halleluja). Your help will be greatly appreciated.

\bigskip
\textit{Salzburg, April 2020\\
Wolfgang Esser-Skala}


\chapter*{Contents.}

%\input{../out/lilypond.toc}

\cleardoublepage
\fi

\iftemplate
\includepdf[pages=-]{../out/\lypdfname.pdf}
\fi



\end{document}