\documentclass[parskip=full]{scrreprt}

\RedeclareSectionCommand[pagestyle=plain,afterskip=10pt plus 1pt]{chapter}
\setkomafont{chapter}{\mdseries\headingfont\fontsize{40}{40}\selectfont\color{black!80}}
\setkomafont{pageheadfoot}{\normalsize}

\def\pnumbox#1{#1\hspace*{9cm}}
\DeclareTOCStyleEntry[
  indent=0pt,
  entrynumberformat=\textcolor{oldred},
  numwidth=2em,
  linefill=\hfill,
  pagenumberbox=\pnumbox,
  pagenumberformat=\itshape
]{tocline}{section}

\usepackage[english]{babel}

\usepackage{fontspec}

\setmainfont{Source Sans Pro}[
  ItalicFont = Source Sans Pro Italic,
  BoldFont = Source Sans Pro Bold,
  BoldItalicFont = Source Sans Pro Bold Italic,
  FontFace = {lt}{n}{Source Sans Pro Light},
  FontFace = {lt}{it}{Source Sans Pro Light Italic},
  FontFace = {sb}{n}{Source Sans Pro Semibold},
  FontFace = {sb}{it}{Source Sans Pro Semibold Italic},
  Numbers = Proportional,
  Ligatures = Common
]
\DeclareRobustCommand{\ltseries}{\fontseries{lt}\selectfont}
\DeclareRobustCommand{\sbseries}{\fontseries{sb}\selectfont}
\DeclareTextFontCommand{\textlt}{\ltseries}
\DeclareTextFontCommand{\textsb}{\sbseries}

\newfontfamily\headingfont{Fredericka the Great}

\usepackage[pass]{geometry}
\newgeometry{twoside,inner=20mm,outer=40mm,top=20mm,bottom=40mm}

\usepackage{url}
\urlstyle{same}

\usepackage{microtype}
\microtypesetup{verbose=silent}

\usepackage{booktabs,array,longtable}
\newcolumntype{L}[1]{>{\raggedright\let\newline\\\arraybackslash\hspace{0pt}}p{#1}}

\usepackage{graphicx}

\usepackage{xcolor}
\definecolor{oldred}{rgb}{.8313,0,0}

\usepackage{pdfpages}

\usepackage{multicol}

\usepackage{scrlayer-scrpage}
\pagestyle{scrheadings}
\clearscrheadfoot
\cfoot[\thepage]{\thepage}
\pagenumbering{roman}


\makeatletter

\newcommand\fancytitlehead{
	\headingfont%
	\fontsize{80}{80}\selectfont\textcolor{black!80}{\@lastname.}\\[15pt]%
	\fontsize{60}{60}\selectfont\@ifundefined{@shorttitle}{\@title}{\@shorttitle}.%
}


\def\firstname#1{\def\@firstname{#1}}
\def\lastname#1{\def\@lastname{#1}}
\def\shorttitle#1{\def\@shorttitle{#1}}
\def\namesuffix#1{\def\@namesuffix{#1}}
\def\instrumentation#1{\def\@instrumentation{#1}}
\def\parts#1{\def\@parts{#1}}

\firstname{\relax}
\lastname{\relax}
\namesuffix{\relax}
\instrumentation{\relax}
\parts{\relax}


\def\maketitle{%
\begin{titlepage}%
	\Large%
	{\@titlehead}%
	\vfill%
	{\strut\@firstname}\\%
	{\sbseries\color{oldred}\strut\@lastname}\\%
	{\strut\@namesuffix}%
	\vfill%
	{\sbseries\@title}\\%
	{\@subtitle}\\[\baselineskip]%
	{\itshape\@instrumentation}%
	\vfill%
	{\itshape\@parts}\hspace*{\fill}\raisebox{0pt}[0pt][0pt]{\includegraphics{ees_logo}}%
\end{titlepage}%
}


\newif\iftemplate\templatetrue
\newif\ifprintreport\printreportfalse
\directlua{
scores = {
  cl1 = "Clarinetto I",
  cl2 = "Clarinetto II",
  fag1 = "Fagotto I",
  fag2 = "Fagotto II",
  ottoni = "Corno I, II\string\\newline Tromba I, II\string\\newline Timpani",
  trb123 = "Trombone I, II, III",
  vl1 = "Violino I",
  vl2 = "Violino II",
  vla = "Viola",
  coro = "Coro",
  org = "Organo",
  b = "Bassi",
  full_score = "Full Score"
}

last_arg = arg[\string#arg]
texio.write("Last argument: " .. last_arg)
if not (scores[last_arg] == nil) then
  tex.print("\string\\def\string\\lypdfname{" .. last_arg .. "}")
  tex.print("\string\\parts{" .. scores[last_arg] .. "}")
  if (last_arg == "full_score") then
    tex.print("\string\\printreporttrue")
  end
end
}

\@ifundefined{lypdfname}{%
  \templatefalse
  \printreporttrue
  \parts{Draft}
}{\templatetrue}

\def\@firstoffour#1#2#3#4{#1}
\def\@secondoffour#1#2#3#4{#2}
\def\@thirdoffour#1#2#3#4{#3}
\def\@lastoffour#1#2#3#4{#4}
\def\lyrefnumber#1{\expandafter\@setref\csname r@#1\endcsname\@firstoffour{#1}}
\def\lyreftitle#1{\expandafter\@setref\csname r@#1\endcsname\@secondoffour{#1}}
\def\lyreftitledansk#1{\expandafter\@setref\csname r@#1\endcsname\@thirdoffour{#1}}
\def\lyrefpage#1{\expandafter\@setref\csname r@#1\endcsname\@lastoffour{#1}}
\def\lyref#1{%
  \begingroup%
  \@tempdima.5\linewidth\relax%
  \advance\@tempdima.5\columnsep\relax%
  \sbseries%
  \makebox[0pt][r]{\color{oldred}\lyrefnumber{#1}\hspace*{1.5em}}%
  \makebox[\@tempdima][l]{\lyreftitle{#1}\Dotfill}%
  \textit{\lyreftitledansk{#1}}%
  \Dotfill\lyrefpage{#1}%
  \endgroup%
}

\def\Dotfill{\leavevmode\cleaders\hb@xt@ .75em{\hss .\hss }\hfill \kern \z@}

\makeatother

\input{../lilypond.ref}




\begin{document}

\titlehead{\fancytitlehead}
\firstname{Friedrich Ludwig Æmilius} % Libretto Jens Immanuel Baggensen
\lastname{Kunzen}
\title{Skabningens Halleluja}
\shorttitle{Skabningens\\Halleluja}
\subtitle{Das Halleluja der Schöpfung\\(DK-Kk mu 6506.1133)}
\instrumentation{2 S, A, T, B (coro), 2 fl, 2 ob, 2 cl, 2 fag, 2 cor, 2 tr, 3 trb, timp,\\vl solo, 2 vl, vla, 2 vlc, b}
\maketitle


\thispagestyle{empty}

\vspace*{\fill}

\raisebox{-4mm}{\includegraphics{byncsaeu}}\hspace*{1em}Wolfgang Esser-Skala, 2020

© 2020 by Wolfgang Esser-Skala. This edition is licensed under the Creative Commons Attribution-NonCommercial-ShareAlike 4.0 International License. To view a copy of this license, visit \url{http://creativecommons.org/licenses/by-nc-sa/4.0/}. 

Music engraving by LilyPond 2.18.0 (\url{http://www.lilypond.org}).\\
Front matter typeset with Source Sans Pro and Fredericka the Great.

\textit{First version, April 2020}

\vspace*{2cm}

\ifprintreport
\chapter*{Critical Report.}

This edition bases upon
\begin{enumerate}
	\item the autograph manuscript in the Det Kongelige Bibliotek på Slotsholmen – Den Sorte Diamant, available at \url{http://img.kb.dk/ma/kunzen/SkaHal_m.pdf} (siglum mu 6506.1133); and
	\item the printed edition (Hans Georg Nägeli, Zürich, 1802) in the Zentralbibliothek Zürich, available at \url{https://doi.org/10.3931/e-rara-24804} (siglum AMG XIV 316).
\end{enumerate}

In general, this edition closely follows the autograph manuscript; the printed edition was mainly used as source for the German lyrics. Any changes that were introduced by the editor are indicated by italic type (dynamics and directions), parentheses (expressive marks) or dashes (slurs and ties). Accidentals are used according to modern conventions. Asterisks denote changes that are clarified in the detailed remarks below.

\bigskip

\footnotetext[1]{Abbreviations: A, alto; b, violoncello and bass; B, basso; cl, clarinet; cor, horn; fag, bassoon; ob, oboe; Ms, manuscript; r, rest; S, soprano; T, tenore; timp, timpani; tr, trumpet; trb, trombone; vl, violin; vla, viola.}

\begin{longtable}{lll L{10cm}}
	\toprule
	\itshape Mov. & \itshape Bar & \itshape Staff & \itshape Note \\
	\midrule \endhead
	–  & –   & org   & All bass figures were supplemented by the editor. \\
	1  & 58  & timp  & 2nd half of bar in Ms: c4–c4 \\
	   & 31  & B     & 4th quarter: grace note missing in Ms \\
	4  & 43  & cor 2 & 4th quarter in Ms: c″4 \\
	5  & 16  & fag 2 & grace note missing in Ms \\
	   & 34  & T     & grace note missing in Ms \\
	   & 44  & vlc   & bar in Ms: G8–d8–f8–e4. \\
	   & 108 & fag 1 & 3rd quarter in Ms: r8–d8 \\
	   & 149 & fag 1 & bar in Ms: r1 \\
	6  & 33  & T, B  & T solo in autograph Ms, B solo in printed edition \\
	\bottomrule
\end{longtable}


This edition has been compiled and checked with utmost diligence. Nevertheless, errors and mistakes cannot be totally excluded. Please report any errors and mistakes to \url{wolfgang@esser-skala.at} or create an issue or pull request on the edition’s GitHub page (https://github.com/skafdasschaf/kunzen-skabningens-halleluja). Your help will be greatly appreciated.

\bigskip
\textit{Salzburg, April 2020\\
Wolfgang Esser-Skala}




\chapter*{Contents. Lyrics.}


\begin{multicols}{2}[\lyref{brichnatur}]
Brich, Natur, in Loblied aus!\\
Lobt Jehoven, alle Zonen,\\
lobet, alle Nationen,\\
lobet Gott!\\
Preiset ihn in vollen Chören,\\[\baselineskip]
Himmel, Erd und Meer!\\
Die ihr athmet, preist den Guten!\\
Ihn, der schuf des Lichtes Gluthen,\\
Ihn, der Odem gab!\\
Hoch in frohe Jubeltöne\\
schweb empor, Gesang, zu Gott!\\
Der gesammten Schöpfung Söhne\\
brechet aus in Jubeltöne,\\
lobet Gott!\\
Der gesammten Schöpfung Söhne\\
singt im heiligsten der Töne,\\
lobet Gott!\columnbreak

\itshape
Bryd, o Støv, i Lovsang ud!\\
Lover Herren alle Zoner!\\
Lover alle Nationer!\\
Lover Gud!\\
Priiser ham, den Eene Store!\\
Priiser Ham i fulde Chore,\\
Himmel, Jord, og Hav!\\
Alt hvad aander, nynne, stamme\\
Ham, som tændte Lysets Flamme!\\
Ham, som Aande gav!\\
Høit i glade Jubeltoner\\
bryde Livets Lovsang [\textit{or}: Vellyst] ud!\\
Alt det Skabtes [\textit{or}: Alle Kloders] Millioner,\\
toner i foreente Toner,\\
toner Gud!
\end{multicols}

\begin{multicols}{2}[\lyref{dieoede}]
Die Öde starrt,\\
von keinem Strahl erhellt,\\
im Schlund der Nacht war Leere, Dunkel, Tod.\\
Da schallt durchs Chaos schaffend dein Gebot:\\
„Sey Welt!“\\
Ein Sonnenheer entsprang aus alter Nacht\\
in Licht hervor.\\
Der Neugeschaffnen Hymne tönt empor:\\
„Halleluja wir leben!\\
Halleluja! du bist! du warst! und bleibest ewig\\
Herr, unser Gott!“\columnbreak

\itshape
Alt hylled laae [\textit{or}: slumred dybt]\\
i Muelighedens Skiød,\\
i Nattens Svælg var Tomhed, Mørke, Død.\\
Men giennem Evighedens Chaos lød\\
dit Bliv!\\
Og Verdener sprang frem af Intets Skiød\\
til Lys og Liv.\\
Nyskabte Væsners Studsen rundt brød ud:\\
„Halleluja! vi ere!\\
Halleluja! du er, du var, og du skal være\\
vor Gud!”
\end{multicols}
\clearpage

\begin{multicols}{2}[\lyref{vomschlaf}]
Vom Schlaf im Schattenwald erwacht,\\
sieht, wenn der Morgen wieder lacht,\\
was lebt, sich deine Herrlichkeit erneun!\\
Sieh! Vögel froh auf leichten Schwingen\\
hinauf in goldne Wolken dringen,\\
ihr Loblied wiederhallt in Flur und Hain!\\
Weit strahlet wachsend deine lichte\\
\hspace*{1em}Morgenröthe!\\
Mild glänzt der Thau, die Rose glüht,\\
sanft weht die Luft, der Bach entflieht,\\
und Haingesang ertönt zur Schöpfungs Feier,\\
milder Thau, Rosenduft,\\
Haingesang, kühle Luft\\
begrüßt den Tag, erweckt zur Morgenfeier.\columnbreak

\itshape
Fra Skovens [\textit{or}: Lundens] skyggefulde Lye\\
seer vækket Liv, hver Morgen nye,\\
i Mark [\textit{or}: Skov] og Eng din Herlighed fremgrye!\\
See! Fuglen fro, paa lette Vinger\\
sig op i Luftens Purpur svinger,\\
Din [\textit{or}: Dens] Lovsang dirrer høit i gyldne Skye!\\
Vidt straaler voxende din lyse\\
\hspace*{1em}Morgenrøde!\\
Og Duggens Blik, og Blomstrets Smiil,\\
og Vindens Vift, og Bækkens Iil,\\
i yndig Kappen stræber den imøde,\\
Duggens Blik, Blomstrets Smiil,\\
Vindens Vift, Bækkens Iil\\
i yndig Kappen stræber den imøde.
\end{multicols}

\begin{multicols}{2}[\lyref{durollest}]
Du rollest auf der Dämmrung Flor!\\
Die Thräne trocknest du,\\
die thauend Blumen tränkte!\\
Dein Himmel graut nicht, wie zuvor –\\
des Tages lichter Held tritt auf in vollem Glanze!\\
Umarmend die blühende Braut,\\
entzückt von des Lichtes belebender Wonne,\\
erzählet die Schöpfung es laut:\\
Das ist Gott!\\
Halleluja! wir leben!\\
Halleluja! du bist, du warst, und bleibest ewig,\\
Herr, unser Gott!\columnbreak

\itshape
Du ruller bort Naturens Slør!\\
Du tørrer Duggens Graad,\\
som Jordens Ansigt væder!\\
Din Himmel skiules ei, som før —\\
see! Dagens lyse Helt i fulde Glands fremtræder!\\
Han favner sin rødmende Brud,\\
og tryllet af Lysets oplivende Glæder\\
den hele Natur bryder ud:\\
Det er Gud!\\
Halleluja! vi ere!\\
Halleluja! du er, du var, og du skal være,\\
vor Gud!
\end{multicols}

\begin{multicols}{2}[\lyref{ichhoerte}]
Ich hörte Haingesang,\\
die Rose that sich auf, das Laub erbebte;\\
ich sah das Morgenroth\\
und fühlte Lust, die neu mein Herz belebte.

Doch einsam war ich, meine Brust\\
fand leer die nicht getheilte Lust.\\
Ich fand die junge Ros im Morgenlicht;\\
ich lächelt ihr; doch sie verstand mich nicht.

Da hörtest Schöpfer du der Sehnsucht Regen;\\
und sieh!\\
Holdlächelnd kam die Gattin mir entgegen!\\
Nun blickt ich froh ins Morgenroth,\\
und rufte: Gut ist Gott!

\columnbreak\itshape
Jeg hørte Lærken slaae,\\
og Rosen aabned sig, og Løvet bæved;\\
jeg Morgenrøden saae,\\
og, vakt til Lyst, mit Hierte høit sig hæved.

Men eene var jeg, og mit Bryst\\
fandt Savn i sin udeelte Lyst.\\
Jeg Roser fandt paa min bestraalte Vei;\\
de rødmede; men de forstod mig ei.

Da saae du, Skaber! mine Taarer rinde;\\
og see!\\
De sendte mig den smilende Mandinde!\\
Jeg sank i Fryd ved Eegens Rod,\\
og raabte: Gud er god!
\end{multicols}

\begin{multicols}{2}
Ich sah der Sterne Heer\\
im festlichen Triumph den Tag begleiten;\\
ich sah der Sonne Sieg,\\
mein Geist verlohr sich in des Himmels\\
\hspace*{1em}Weiten.

Doch einsam war ich, meine Brust\\
fand Wehmut nur in halber Lust.\\
Ich rief den Bäumen, sich mit mir zu freun;\\
doch mich verstand kein Baum im ganzen\\
\hspace*{1em}Hain.

Da hörtest du der Thräne leises Flehen;\\
und sieh!\\
Mein ander Ich trat her von jenen Höhen –\\
O! Gut ist Gott! rief er mir zu,\\
ich rufte: Gott ist gut!

Ich sah des Morgens Purpurzelt –\\
ich sahs, er kam, des Tages Held –\\
doch was ist, Morgenroth, dein Blick?\\
Wie weicht der Sonne Gluth zurück?\\
Was bist du schönes Thal –\\
was bist du hoher Himmel –\\
vorm Feuerblick des Manns?\\
Ach, vor der Gattin Huld?

O, was ist aller Herrlichkeiten Fülle\\
vor dir, der Liebe Gluth?\\
Vor unsrer Herzen Lust?

Preis, Liebe, dir, wir leben!\\
Preis, Liebe, dir! du bist, du warst\\
und bleibest ewig,\\
o Liebe, unsre Lust!

\columnbreak\itshape
Jeg Lysets Hære saae\\
med Dagen i Triumph til Jorden stimle;\\
jeg Solen seire saae,\\
og Tanken tabte sig i Himles\\
\hspace*{1em}Himle.

Men eene var jeg, og mit Bryst\\
fandt Veemod i sin halve Lyst.\\
Jeg iilte længselfuld hvert Træ imod;\\
men intet Træ i Lunden mig\\
\hspace*{1em}forstod.

Da saae du, Fader! Taaren i mit Øye;\\
og see!\\
Min anden Jeg kom frem bag Skovens Høye —\\
O! Gud er god! saa brød han ud,\\
jeg svarte: God er Gud!

Jeg Østens lyse Purpur saae —\\
jeg skued Dagens Helt fremgaae —\\
men hvad er Morgenrødens Smiil?\\
Men hvad er Solens Straalers Ild?\\
Hvad er du, skiønne Jord —\\
hvad er du, høie Himmel —\\
mod Mandens Tankeblik?\\
Imod Mandindens Smiil?

O hvad er alle Herligheders Vrimmel\\
mod Flammen i vort Bryst?\\
Mod vore Hierters Lyst?

O Kierlighed! vi ere!\\
O Kierlighed! du er, du var,\\
og du skal være\\
vor Lyst!
\end{multicols}


\begin{multicols}{2}[\lyref{wirpreisen}]
Wir preisen dich, wir danken dir, o Gott!\\
Im Glanz des Lichts, in stiller Freude Schooß.

Doch sehn wir deinen Wink und hören\\
\hspace*{1em}dein Geboth\\
in schwarzer Nacht, wo Schreckenston\\
\hspace*{1em}uns dräut.

Dein Thron wird Nacht, die Sonne fleucht\\
vom Wolkenzug verscheucht;\\
des Abgrunds Mächte sich erheben;

\columnbreak\itshape
Vi love dig, vi takke dig, o Gud!\\
I Lysets Glands, i Livets stille Fryd.

Men og vi see dit Vink og høre\\
\hspace*{1em}dine Bud\\
i Nattens Mulm, og Skrækkens døve\\
\hspace*{1em}Lyd.

Din Throne mørknes, Solen flyer\\
bag tykke Skyer;\\
sig Dybets sorte Kæmpe hæver;
\end{multicols}


\begin{multicols}{2}
vom Donnerschlag der Erde Festen beben.

Sieh, Thäler schwellen! Höhen stürzen ein!\\
Wild zischen Flammen hinter Wassern drein!\\
Der Sturm durchsaust zersprengter Berge\\
\hspace*{1em}Spalten,\\
wo Gluthen über Lavaströmen walten!\\
Mit Schlag auf Schlag, mit Knall auf Knall,\\
im Sturz zerschlagner Felsen, Klippenfall,\\
mit schwarzer Wetter schrecklichem Getümmel\\
rollt dumpf dein Donnerwagen durch die\\
\hspace*{1em}Himmel!

Gerechter Gott, du furchtbar mächtger Gott!\\
Wir tief im Staube hören dein Geboth.

\columnbreak\itshape
og Jordens lynildslagne Grundvold bæver.

See, Dale svulme! Høie synke ned!\\
Vildt flammer Ilden hen i Vandets Fied!\\
Igiennem vidtadsplittede\\
\hspace*{1em}Ruiner\\
i høit opslagne Luer Stormen hviner!\\
Med Brag i Brag, og Skrald i Skrald,\\
i knuste Fieldes Styrten, Klippers Fald,\\
med bange Dundrens hule Rædselbulder\\
din Dommervogn igiennem Himlen\\
\hspace*{1em}rulder!

Retfærdige, forfærdelige Gud!\\
Vi dybt i Støvet høre dine Bud.
\end{multicols}

\begin{multicols}{2}[\lyref{selbstwenn}]
Selbst wenn des Lebens Engel alle flüchten,\\
der Tod auf deinen Wink entsteigt des Abgrunds\\
\hspace*{1em}Schlüchten,\\
beim ernsten Gang zur Grabesruh\\
bist noch des Herzens letzter Seufzer du.

Ja, Vater! selbst im Tod, im bangen Nu\\
\hspace*{0pt}\\
soll aus dem Staub zu dir der Geist noch flehen,\\
zu dir hinauf der Wurm im Moder sehen,\\
denn sein, wie deines Serafs, denkest du.

\columnbreak\itshape
Self [or: Men] naar sig alle Livets Engle skiule,\\
naar Døden paa dit Vink sig reiser af sin\\
\hspace*{1em}Hule,\\
i Morderslagets bange Nu\\
vort knuste Hiertes sidste Suk er du.

Ja, Fader! selv i Dødens bange Nu\\
\hspace*{1em}[or: skumle Gru]\\
skal fra dit Støv vort Intet dig anraabe,\\
og Ormen i sit Muld med Tillid haabe,\\
du kommer den, som Seraph selv, ihu.
\end{multicols}

\begin{multicols}{2}[\lyref{gerechter}]
Gerechter Richter! du bist ewig gut!\\
Du gabst uns Glauben, Liebe und Vertrauen,\\
wir dürfen fest auf deine Hülfe bauen,\\
denn schützend waltet ob uns deine Hut.\\
Du siehst herab mit Vaterfreundlichkeit,\\
wenn kindlich froh des Lebens Bahn wir wallen;\\
in Eintracht schenkst du deinen Kindern allen\\
den Vorschmack von des Himmels\\
\hspace*{1em}Seeligkeit.

\columnbreak\itshape
Thi, store Dommer! du er evig god!\\
Du gav vort Hierte Kiærlighedens Vinge,\\
vi trygge dig vor Tillids Offer bringe,\\
du tager det med Velbehag imod.\\
Du seer med milde Faderblikke ned,\\
naar barnlig fro vi Livets Bane vandre;\\
du gav os her i Samfund med hverandre\\
en Forsmag af Seraphers\\
\hspace*{1em}Salighed.
\end{multicols}

\clearpage

\begin{multicols}{2}[\lyref{ogott}]
O Gott!\\
Laß schwinden, was du schufst auf dein Gebot!\\
Laß Sterne fallen, Sonnen untergehn!\\
Des Chaos Schlünde offen widerstehn!\\
Ob Himmel, Erd und Meere stürzten ein,\\
Unendlicher, du warst und du wirst seyn!\\
Allstets, im Chaos, Dunkel, fern und nah,\\
bist du!

\columnbreak\itshape
O Gud!\\
Lad dine Hænders Gierning slettes ud!\\
Lad Stjerner falde, Sole svinde hen!\\
Og Mørkets Afgrund aabne sig igien!\\
Om Himmel, Jord, og Hav, og alt forgaaer,\\
du, Evige den samme, du bestaaer!\\
I Alt, i Nattens Øde, fiern og nær,\\
du er! %130
\end{multicols}

\begin{multicols}{2}[\lyref{undleben}]
Und Leben, Licht und Freud ist ewig, so wie du.\\
O Wonn in unsrer Brust!\\
Dir, Liebe, Preis! wir leben!\\
Halleluja! du bist, du warst, und bleibest ewig,\\
o Liebe, unsre Lust!

\columnbreak\itshape
Og Lys og Liv og Lyst er evig hvor du er.\\
O Himmel i vort Bryst!\\
O Kiærlighed! vi ere!\\
Halleluja! du er, du var, og du skal være\\
Vor Lyst!
\end{multicols}

\begin{multicols}{2}[\lyref{heilig}]
Heilig! Heilig! Heilig!\\
Sink, Schöpfung, in den Staub!\\
Neig dich, o Himmel, Erde knie, bet an!

\columnbreak\itshape
Hellig! Hellig! Hellig!\\
Fald hele Skabning ned!\\
Nedbøi dig, Himmel, knæl, o Jord, tilbed!
\end{multicols}

\cleardoublepage
\fi

\iftemplate
\includepdf[pages=-]{../out/\lypdfname.pdf}
\fi



\end{document}